
%%%%%%%%%%%%%%%%%%%%%%% file typeinst.tex %%%%%%%%%%%%%%%%%%%%%%%%%
%
% This is the LaTeX source for the instructions to authors using
% the LaTeX document class 'llncs.cls' for contributions to
% the Lecture Notes in Computer Sciences series.
% http://www.springer.com/lncs       Springer Heidelberg 2006/05/04
%
% It may be used as a template for your own input - copy it
% to a new file with a new name and use it as the basis
% for your article.
%
% NB: the document class 'llncs' has its own and detailed documentation, see
% ftp://ftp.springer.de/data/pubftp/pub/tex/latex/llncs/latex2e/llncsdoc.pdf
%
%%%%%%%%%%%%%%%%%%%%%%%%%%%%%%%%%%%%%%%%%%%%%%%%%%%%%%%%%%%%%%%%%%%


\documentclass[runningheads,a4paper]{llncs}

\usepackage{amssymb}
\setcounter{tocdepth}{3}
\usepackage{graphicx}

\usepackage{url}
\urldef{\mailsa}\path|{alfred.hofmann, ursula.barth, ingrid.haas, frank.holzwarth,|
\urldef{\mailsb}\path|anna.kramer, leonie.kunz, christine.reiss, nicole.sator,|
\urldef{\mailsc}\path|erika.siebert-cole, peter.strasser, lncs}@springer.com|    
\newcommand{\keywords}[1]{\par\addvspace\baselineskip
\noindent\keywordname\enspace\ignorespaces#1}

\begin{document}

\mainmatter  % start of an individual contribution

% first the title is needed
\title{Semantic Annotation Semantically: Using a~Sharable Extraction Ontology and a Reasoner}

% a short form should be given in case it is too long for the running head
\titlerunning{Semantic Annotation Semantically}
%\titlerunning{Towards Semantic Annotation Supported by Dependency Linguistics and ILP}

% the name(s) of the author(s) follow(s) next
%
% NB: Chinese authors should write their first names(s) in front of
% their surnames. This ensures that the names appear correctly in
% the running heads and the author index.
%
\author{Jan D\v{e}dek \and Peter Vojt\'{a}\v{s}}
%\thanks{Please note that the LNCS Editorial assumes that all authors have used
%the western naming convention, with given names preceding surnames. This determines
%the structure of the names in the running heads and the author index.}%
%
\authorrunning{Jan D\v{e}dek \and Peter Vojt\'{a}\v{s}}
% (feature abused for this document to repeat the title also on left hand pages)

% the affiliations are given next; don't give your e-mail address
% unless you accept that it will be published
\institute{Department of Software Engineering, Charles University,\\
Prague, Czech Republic
\\\url{{dedek,vojtas}@ksi.mff.cuni.cz}}

%
% NB: a more complex sample for affiliations and the mapping to the
% corresponding authors can be found in the file "llncs.dem"
% (search for the string "\mainmatter" where a contribution starts).
% "llncs.dem" accompanies the document class "llncs.cls".
%

\toctitle{Semantic Annotation Semantically: Using a Sharable Extraction Ontology and a Reasoner}
\tocauthor{Jan D\v{e}dek and Peter Vojt\'{a}\v{s}}
\maketitle


\begin{abstract}
In this paper we present a method for semantic annotation of texts, which is based on a deep linguistic analysis (DLA) and Inductive Logic Programming xxxxxxxxxxxxxxxxxxxxxxxx

A description, implementation and initial evaluation of the method are the main contributions of the paper.
\keywords{Semantic Annotation, Dependency Linguistics, Inductive Logic Programming, Information Extraction, Machine Learning}
\end{abstract}


\section{Introduction}



\section{Related Work}

\section{Conclusion and Future Work}

\bigskip
\noindent\textbf{Acknowledgments}\\
This work was partially supported by Czech projects: GACR P202/10/0761, GACR-201/09/H057, GAUK 31009 and MSM-0021620838.
The author would like to thank his supervisor Peter Vojt\'{a}\v{s} for the guidance of the PhD thesis.




\bibliographystyle{splncs03}
\bibliography{DedekVojtas_ESWC2010_semantically}
\end{document}
