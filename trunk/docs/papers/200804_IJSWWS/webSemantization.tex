%%%%%%%%%%%%%%%%%%%%%%%%%%%%%%%%%%%%%%%%
%                                      %
% Athanassios Protopapas, October 2005 %
% Mini-example for using apa.cls       %
%                                      %
%%%%%%%%%%%%%%%%%%%%%%%%%%%%%%%%%%%%%%%%

\documentclass[man]{apa}
\usepackage[cp1250]{inputenc}
\usepackage{times,epsfig}



\title{Web semantization}
\author{Peter Vojt�, Jan D�dek, Alan Eckhardt}
\affiliation{Department of software engineering, Faculty of Mathematics and Physics, Charles University\\
Institute of Computer Science, Academy of Sciences of the Czech Republic }

\abstract{Abstract.}

\acknowledgements{.}

\shorttitle{Web semantization}
\rightheader{Web semantization}
\leftheader{P.\ Vojt�}

\begin{document}
\maketitle                            
Introduction :
Here goes the text of the article. Note that the content begins immediately after \texttt{maketitle} and there is no blank line between the title command and the article text. This first section of the article is typically the introduction and, according to APA style, should not bear a section heading.\footnote{That is, there is no ``Introduction'' section.} Subsequent sections, however, are titled according to the psychological conventions.

\section{Problem} 
\subsubsection{Searching a used car}
- We want to buy a used car.
- We don't want to access several web pages. We want all information on one place.
- The decision which car to buy is based on attributes of 
			- the car
				- mark 
				- speed etc.
				- 
			- the seller
				- location
				- price
			
\subsubsection{Keyword search limitations}

\subsection{My Idea}
\subsubsection{Agent proposal}
\subsubsection{Ontologies}
\subsubsection{Web search/LG}
\subsubsection{Decathlon - conflicting objectives/AE}
\subsubsection{Linguistic}

\section{Details of my idea}
\section{Related work}

\section{Conclusion and future work}
\bibliography{citace}

\end{document}
