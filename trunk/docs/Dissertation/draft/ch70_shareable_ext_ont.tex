\chapter{Shareable Extraction Ontologies} \label{ch:Shareable_Extraction_Ontologies}

\graphicspath{{../img/ch70/}}



%%%%%%%%%%%%%%%%%%%%%%%%%%%%%%%%%%%%%%%%%%%%%%%%%%%%%%%%%%%%%%%%%%%%%%%%%%%%%%%%%%%%%%%%%%%%%%%%%%%%%%%%%%%%%%
\section{Introduction} \label{sec:ch70_intro}
%%%%%%%%%%%%%%%%%%%%%%%%%%%%%%%%%%%%%%%%%%%%%%%%%%%%%%%%%%%%%%%%%%%%%%%%%%%%%%%%%%%%%%%%%%%%%%%%%%%%%%%%%%%%%%

The only way how to use such extraction ontology is within the corresponding extraction tool. It is not necessary to have the ontology in a ``owl or rdf file''. In a sense such extraction ontology is just a configuration file. For example in \citep{springerlink:10.1007/978-3-642-01891-6_5} %[Labsky]
 (and also in \citep{DBLP:conf/er/EmbleyTL02}) the so called extraction ontologies are kept in XML files with a proprietary structure and it is absolutely sufficient, there is no need to treat them differently.







%In time it will turn out if such extraction ontologies are useful or not. But for sure they bring something new that was not possible before.


%%%%%%%%%%%%%%%%%%%%%%%%%%%%%%%%%%%%%%%%%%%%%%%%%%%%%%%%%%%%%%%%%%%%%%%%%%%%%%%%%%%%%%%%%%%%%%%%%%%%%%%%%%%%%%
\section{Semantic Annotation Semantically} \label{sec:main}
%%%%%%%%%%%%%%%%%%%%%%%%%%%%%%%%%%%%%%%%%%%%%%%%%%%%%%%%%%%%%%%%%%%%%%%%%%%%%%%%%%%%%%%%%%%%%%%%%%%%%%%%%%%%%%

The problem of extraction ontologies that are not shareable was pointed out in the introduction (Section~\ref{sec:ch70_intro}). 

And this is where the title of the chapter and the present section comes from. 




%%%%%%%%%%%%%%%%%%%%%%%%%%%%%%%%%%%%%%%%%%%%%%%%%%%%%%%%%%%%%%%%%%%%%%%%%%%%%%%%%%%%%%%%%%%%%%%%%%%%%%%%%%%%%%
\section{Experiment} \label{sec:ch70_experiment}
%%%%%%%%%%%%%%%%%%%%%%%%%%%%%%%%%%%%%%%%%%%%%%%%%%%%%%%%%%%%%%%%%%%%%%%%%%%%%%%%%%%%%%%%%%%%%%%%%%%%%%%%%%%%%%


In this section we present an experiment that should serve as a proof of a concept that the proposed idea of independent extraction ontologies is realizable. We have selected several reasoners (namely Jena, HermiT, Pellet and FaCT++) and tested them on two slightly different datasets from two different domains and languages (see Table~\ref{tab:datasets}). This should at least partially demonstrate the universality of the proposed approach.

In both cases the task is to find all instances (corresponding to words in a document) that should be uncovered by the extraction rules. The extraction rules are saved in single extraction ontology for each dataset. The datasets are divided into individual document ontologies (owl files) corresponding to the individual documents. During the experiment the individual document ontologies are processed separately (one ontology in a step) by a selected reasoner. The total time taken to process all document ontologies of a dataset is the measured result of the reasoner for the dataset.

The actual reasoning tasks are more difficult than a simple retrieval of all facts entailed by the extraction rules. Such simple retrieval task took only a few seconds for the Acquisitions v1.1 dataset (including parsing) in the native Prolog environment that the IE engine uses. There were several more inferences needed in the reasoning tasks because the schema of the input files was a little bit different from the schema used in rules. The mapping of the schemas was captured in another ``mapping'' ontology that was included in the reasoning. The mapping ontology is a part of the publically available project ontologies.
%\footnote{See ``Data/ontologies'' link on the project page \url{http://czsem.berlios.de/}}%and a potentially interested reader can find the complete mapping ontology on the project's web site.

\subsubsection{How to Download}
All the resources (including source codes of the case study and the experiment, datasets and ontologies) mentioned in this chapter are publically available on the project's web site\footnote{\url{http://czsem.berlios.de/}} and detailed information can be found there.

\subsection{Datasets} \label{sec:ch70_datasets}

In the experiment we used two slightly different datasets from two different domains and languages.  Table~\ref{tab:datasets} summarizes some basic information about them.

\begin{table}
\begin{center}
\begin{tabular}{|r||l|l|b{20mm}|b{20mm}|b{20mm}|}
%\begin{tabular}{|r||r|r|c|r|c|}
\hline
dataset & domain & language & number of~files &  dataset size (MB) &  number of~rules  \\
\hline
\hline
\textbf{czech\_fireman} & accidents & Czech &  50 &  16 &  2\\
\hline
\textbf{acquisitions} & finance & English &  600 &  126 &  113\\
\hline
\end{tabular}
\caption{Description of datasets that were used.}\centering
\end{center}
\label{tab:datasets}
\end{table}

\subsubsection{Czech Fireman}

\ref{sec:ch40_fireman_annotated} resp. \ref{sec:ch40_rdf_fireman}


The fist dataset is called `czech\_fireman'. This dataset was created by ourselves during the development of our IE  engine. It is a collection of 50 Czech texts that are reporting on some accidents (car accidents and other actions of fire rescue services). These reports come from the web of Fire rescue service of Czech Republic.
%\footnote{\url{http://www.hzscr.cz/hasicien/}}.
%The labeled corpus is publically available on the website of our project\footnote{\url{http://czsem.berlios.de/}}.
The corpus is structured such that each document represents one event (accident) and several attributes of the accident are marked in text. For the experiment we selected the `damage' task -- to find an amount (in CZK - Czech Crowns) of summarized damage arisen during a reported accident.





\subsubsection{Acquisitions v1.1}  

\ref{sec:ch40_corporate_acquisitions} resp. \ref{sec:ch40_rdf_acquisitions}

The second dataset is called ``Corporate Acquisition Events'' and it is
described in \citep{lewis1992representation}. More precisely we used the \emph{Acquisitions v1.1} version\footnote{This version of the corpus comes from the Dot.kom (Designing infOrmation extracTion for KnOwledge Management) project's resources: \url{http://nlp.shef.ac.uk/dot.kom/resources.html}} of the corpus.
This is a collection of 600 news articles describing acquisition
events taken from the Reuters dataset. News articles are tagged to identify fields
related to acquisition events. These fields include `purchaser' , `acquired', and
`seller' companies along with their abbreviated names (`purchabr', `acqabr' and
`sellerabr'). Some news articles also mention the field `deal amount'.


For the experiment we selected only the `acquired' task.





\subsection{Reasoners}

%In the experiment we used four OWL reasoners
Four OWL reasoners were used in the experiment(namely
Jena\footnote{\url{http://jena.sourceforge.net}}
,HermiT\footnote{\url{http://hermit-reasoner.com}}
,Pellet\footnote{\url{http://clarkparsia.com/pellet}}
and FaCT++\footnote{\url{http://code.google.com/p/factplusplus}}
) and the time they spent on processing a particular dataset was measured.The time also includes time spent on parsing the input. HermiT, Pellet and FaCT++ were called through OWL API-3.1, so the same parser was used for them. Jena reasoner was used in its native environment with the Jena parser.
In the early beginning of the experiment we had to exclude the FaCT++ reasoner from both tests. It turned out that FaCT++ does not work with rules\footnote{\url{http://en.wikipedia.org/wiki/Semantic_reasoner#Reasoner_comparison}} and it did not return any result instances.  All the remaining reasoners strictly agreed on the results and returned the same sets of instances.

Also HermiT was not fully evaluated on the Acquisitions v1.1 dataset because it was too slow. The reasoner spent 13 hours of running to process only 30 of 600 files of the dataset. And it did not seem useful to let it continue.












\subsection{Evaluation Results of the Experiment}






\begin{table}
\begin{center}
\begin{tabular}{|r||r|r||r|r|}
\hline
reasoner & \textbf{czech\_fireman} & stdev & \textbf{acquisitions-v1.1} & stdev\\
\hline
\hline
\textbf{Jena} & 161 s & 0.226 & 1259 s & 3.579\\
\hline
\textbf{HermiT} & 219 s & 1.636 & $\gg$ 13 hours & \\
\hline
\textbf{Pellet} & 11 s & 0.062 & 503 s & 4.145\\
\hline
\textbf{FaCT++} & \multicolumn{4}{|c|}{Does not support rules.}\\
\hline
\end{tabular}
\end{center}

Time is measured in seconds. Average values from 6 measurements. Experiment environment: Intel Core I7-920 CPU 2.67GHz, 3GB of RAM, Java SE 1.6.0\_03, Windows XP.

\caption{Time performance of tested reasoners on both datasets.}
\label{tab:results}
\end{table}

Table~\ref{tab:results} summarizes results of the experiment. The standard deviations are relatively small when compared to the differences between the average times.  So there is no doubt about the order of the tested reasoners. Pellet performed the best and HermiT was the slowest amongst the tested and usable reasoners in this experiment.

From the results we can conclude that similar tasks can be satisfactorily solved by contemporary OWL reasoners because three of four tested reasoners were working correctly and two reasoners finished in bearable time.

On the other hand even the fastest system took 8.5 minutes to process 113 rules over 126MB of data. This is clearly   significantly longer than a bespoke system would require. 
Contemporary Semantic Web reasoners are known still to be often quite inefficient and the experiment showed that using them today to do information extraction will result in quite poor performance. However, efficiency problems can be solved
and in the context of Linked Data providing shareable descriptions of information extraction rules may be valuable.





%\subsection{Repeatability}
%
%Our implementation is publicly available -- source codes and the datasets can be downloaded from our project's web-page\footnote{\url{http://czsem.berlios.de/}}, so it should be also possible to repeat the experiment in a sense of  the SIGMOD Experimental Repeatability Requirements \citep{biblio:SIGMODrepeatability}.
%%%%%%%%%%%%%%%%%%%%%%%%%%%%%%%%%%%%%%%%%%%%%%%%%%%%%%%%%%%%%%%%%%%%%%%%%%%%%%%%%%%%%%%%%%%%%%%%%%%%%%%%%%%%%%
\section{Discussion} \label{sec:discuss}
%%%%%%%%%%%%%%%%%%%%%%%%%%%%%%%%%%%%%%%%%%%%%%%%%%%%%%%%%%%%%%%%%%%%%%%%%%%%%%%%%%%%%%%%%%%%%%%%%%%%%%%%%%%%%%


In this chapter (Section~\ref{sec:ch70_doc_ont}) we have described a method how to apply an extraction ontology to a document ontology and obtain so called ``annotated'' document ontology. To have an ``annotated'' document ontology is almost the same as to have an annotated document. An annotated document is useful (easier navigation, faster reading and lookup of information, possibility of structured queries on collections of such documents, etc.) but if we are interested in the actual information present in the document, if we want to know the facts that are in a document asserted about the real word things then an annotated document is not sufficient. But the conversion of an annotated document to the real world facts is not simple.
There are obvious issues concerning data integration and duplicity of information. For example when in a document two mentions of people are annotated as `injured', what is then the number of injured people in the corresponding accident? Are the two annotations in fact linked to the same person or not?

In the beginning of our work on the idea of shareable extraction ontologies we planned to develop it further, we wanted to cover also the step from annotated document ontologies to the real world facts. The extraction process would then end up with so called ``fact ontologies''. But two main obstacles prevent us to do that.

\begin{enumerate}
	\item Our IE engine is not yet capable to solve these data integration and duplicity of information issues and the real world facts would be quite imprecise then.
	\item There are also technology problems of creating new facts (individuals) during reasoning.
\end{enumerate}

Because of the decidability and finality constraints of the Description Logic Reasoning it is not possible to create new individuals during the reasoning process. There is no standard way how to do it. But there are some proprietary solutions like \verb@swrlx:createOWLThing@\footnote{\url{http://protege.cim3.net/cgi-bin/wiki.pl?action=browse&id=SWRLExtensionsBuiltIns}} from the Prot\'{e}g\'{e} project and \verb@makeTemp(?x)@ or \verb@makeInstance(?x, ?p, ?v)@\footnote{\url{http://jena.sourceforge.net/inference/#RULEbuiltins}} from the Jena project.
And these solutions can be used in the future work. 

\subsection{SPARQL Queries -- Increasing Performance?}

There is also a possibility to transform the extraction rules to SPARQL construct queries. This would probably rapidly increase the time performance. However a document ontology would then have to exactly fit with the schema of the extraction rules.  This would be a minor problem. 

The reason why we did not study this approach from the beginning is that we were interested in extraction \emph{ontologies} and SPARQL queries are not currently regarded as a part of an ontology and nothing is suggesting it to be other way round.  

Anyway the performance comparison remains a valuable task for the future work.

\subsection{Contributions for Information Extraction}The chapter combines the field of ontology-based information extraction and rule-based reasoning. The aim is to show a new possibility in usage of IE tools and reasoners. In this chapter we do not present a solution that would improve the performance of IE tools.
We also do not provide a proposal of a universal extraction format (although a specific form for the rule based extraction on dependency parsed text could be inferred). This task is left for the future if a need for such activity emerges.
%The aim of the chapter is a demonstration of the idea of tool independent extraction ontologies and the possibility to use reasoners for information extraction.


%%%%%%%%%%%%%%%%%%%%%%%%%%%%%%%%%%%%%%%%%%%%%%%%%%%%%%%%%%%%%%%%%%%%%%%%%%%%%%%%%%%%%%%%%%%%%%%%%%%%%%%%%%%%%%
\section{Conclusion} \label{sec:ch70_conclusion}
%%%%%%%%%%%%%%%%%%%%%%%%%%%%%%%%%%%%%%%%%%%%%%%%%%%%%%%%%%%%%%%%%%%%%%%%%%%%%%%%%%%%%%%%%%%%%%%%%%%%%%%%%%%%%%

%In the end of the chapter we would like to summarize the main contributions of the chapter.

%\begin{itemize}
	%\item In the beginning of the chapter we pointed out the draw back of so called extraction ontologies -- in most cases they are dependent on a particular extraction/annotation tool and they cannot be used separately.	
	%\item We extended the concept of extraction ontologies by adding the shareable aspect and we introduced a new principle of making extraction ontologies independent of the original tool: the possibility of application of an extraction ontology to a document by an ordinary reasoner.
	%\item In Section~\ref{sec:case} we presented a case study that shows that the idea of shareable extraction ontologies is realizable. We presented implementation of an IE tool that exports its extraction rules to an extraction ontology and we demonstrated how this extraction ontology can be applied to a document by a reasoner.
	%\item Moreover in Section~\ref{sec:ch70_experiment} an experiment with several OWL reasoners was presented. The experiment evaluated the performance of contemporary OWL reasoners on IE tasks (application of extraction ontologies).  
	%\item A new publically available benchmark for OWL reasoning was created together with the experiment. Other reasoners can be tested this way.
%\end{itemize}
   
In the beginning of the chapter we pointed out the draw back of so called extraction ontologies -- in most cases they are dependent on a particular extraction/annotation tool and they cannot be used separately.	

We extended the concept of extraction ontologies by adding the shareable aspect and we introduced a new principle of making extraction ontologies independent of the original tool: the possibility of application of an extraction ontology to a document by an ordinary reasoner.

In Section~\ref{sec:case} we presented a case study that shows that the idea of shareable extraction ontologies is realizable. We presented implementation of an IE tool that exports its extraction rules to an extraction ontology and we demonstrated how this extraction ontology can be applied to a document by a reasoner.

Moreover in Section~\ref{sec:ch70_experiment} an experiment with several OWL reasoners was presented. The experiment evaluated the performance of contemporary OWL reasoners on IE tasks (application of extraction ontologies). A new publically available benchmark for OWL reasoning was created together with the experiment. Other reasoners can be tested this way.

%We would like to conclude the chapter by stating that only time will show if the fundamental idea of the chapter will be useful but today it is at least a new use case for both: usage of IE tools and reasoners.

