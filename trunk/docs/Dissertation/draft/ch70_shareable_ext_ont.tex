\chapter{Shareable Extraction Ontologies} \label{ch:Shareable_Extraction_Ontologies}

\graphicspath{{../img/ch70/}}



%%%%%%%%%%%%%%%%%%%%%%%%%%%%%%%%%%%%%%%%%%%%%%%%%%%%%%%%%%%%%%%%%%%%%%%%%%%%%%%%%%%%%%%%%%%%%%%%%%%%%%%%%%%%%%
\section{Introduction} \label{sec:ch70_intro}
%%%%%%%%%%%%%%%%%%%%%%%%%%%%%%%%%%%%%%%%%%%%%%%%%%%%%%%%%%%%%%%%%%%%%%%%%%%%%%%%%%%%%%%%%%%%%%%%%%%%%%%%%%%%%%

The only way how to use such extraction ontology is within the corresponding extraction tool. It is not necessary to have the ontology in a ``owl or rdf file''. In a sense such extraction ontology is just a configuration file. For example in \citep{springerlink:10.1007/978-3-642-01891-6_5} %[Labsky]
 (and also in \citep{DBLP:conf/er/EmbleyTL02}) the so called extraction ontologies are kept in XML files with a proprietary structure and it is absolutely sufficient, there is no need to treat them differently.







%In time it will turn out if such extraction ontologies are useful or not. But for sure they bring something new that was not possible before.


%%%%%%%%%%%%%%%%%%%%%%%%%%%%%%%%%%%%%%%%%%%%%%%%%%%%%%%%%%%%%%%%%%%%%%%%%%%%%%%%%%%%%%%%%%%%%%%%%%%%%%%%%%%%%%
\section{Semantic Annotation Semantically} \label{sec:main}
%%%%%%%%%%%%%%%%%%%%%%%%%%%%%%%%%%%%%%%%%%%%%%%%%%%%%%%%%%%%%%%%%%%%%%%%%%%%%%%%%%%%%%%%%%%%%%%%%%%%%%%%%%%%%%

The problem of extraction ontologies that are not shareable was pointed out in the introduction (Section~\ref{sec:ch70_intro}). 

And this is where the title of the chapter and the present section comes from. 




%%%%%%%%%%%%%%%%%%%%%%%%%%%%%%%%%%%%%%%%%%%%%%%%%%%%%%%%%%%%%%%%%%%%%%%%%%%%%%%%%%%%%%%%%%%%%%%%%%%%%%%%%%%%%%
%\section{Experiment} \label{sec:ch70_experiment}
%%%%%%%%%%%%%%%%%%%%%%%%%%%%%%%%%%%%%%%%%%%%%%%%%%%%%%%%%%%%%%%%%%%%%%%%%%%%%%%%%%%%%%%%%%%%%%%%%%%%%%%%%%%%%%






%\subsection{Repeatability}
%
%Our implementation is publicly available -- source codes and the datasets can be downloaded from our project's web-page\footnote{\url{http://czsem.berlios.de/}}, so it should be also possible to repeat the experiment in a sense of  the SIGMOD Experimental Repeatability Requirements \citep{biblio:SIGMODrepeatability}.
