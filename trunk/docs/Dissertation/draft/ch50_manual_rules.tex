\chapter{Extraction Method Based on Manually Created Rules} \label{sec:ch50_manual_rules_chapter}
\graphicspath{{../img/ch50/}}

%%%%%%%%%%%%%%%%%%%%%%%%%%%%%%%%%%%%%%%%%%%%%%%%%%%%%%%%%%%%%%%%%%%%%%%%%%%%%%%%%%%%%
\section{Introduction}
%%%%%%%%%%%%%%%%%%%%%%%%%%%%%%%%%%%%%%%%%%%%%%%%%%%%%%%%%%%%%%%%%%%%%%%%%%%%%%%%%%%%%

\subsection{Presented Extraction Methods}

The main motivation for creating both these methods was an attempt to use deep linguistic analysis of natural language texts. Especially for the Czech language with free word order this seemed reasonable. It is much more straightforward to design extraction rules on the basis of linguistic dependency trees than to struggle with the surface structure of text. In a dependency tree a position of a word is determined by its syntactic (analytical trees) or even semantic role (tectogrammatical trees). So the extraction rules might not be dramatically affected by minor variations (not changing the factual meaning of a sentence) of the word order.







%%%%%%%%%%%%%%%%%%%%%%%%%%%%%%%%%%%%%%%%%%%%%%%%%%%%%%%%%%%%%%%%%%%%%%%%%%%%%%%%%%%%%
\section{Conclusion}
%%%%%%%%%%%%%%%%%%%%%%%%%%%%%%%%%%%%%%%%%%%%%%%%%%%%%%%%%%%%%%%%%%%%%%%%%%%%%%%%%%%%%

Note that in this case we did not concern the annotation aspect of information extraction. Although it is possible to infer an annotation based variety of the presented method, here we did not take it into account because during the development of the method the aim was more to produce structured data from text than to produce annotated documents. In the next section the emphasis will be inverted. 

A deeper evaluation of the method would be definitely interesting, but at the moment the information provided is the only available. There is no real world application of the method outside the academic ground. The method is still waiting for deep testing and further development in an extensive real world project.

