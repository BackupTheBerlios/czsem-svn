\chapter{Extraction Method Based on ILP Machine Learning} \label{ch:ILP_Learning}
\graphicspath{{../img/ch60/}}


\section{Introduction}
Automated semantic annotation (SA) is considered to be one of the most important elements in the evolution of the Semantic Web. Besides that, SA can provide great help in the process of data and information integration and it could also be a basis for intelligent search and navigation.


A description, implementation and initial evaluation of the method are the main contributions of the present work.







%\section{Data Flow ?or Schema of the Extraction Process?}






\section{How to Download}
The project website\footnote{\url{http://czsem.berlios.de}} provides several ways how to get all the presented tools running. A platform independent installer, Java binaries and source codes are provided under the GPL license.


\section{Evaluation}
\label{sec:evaluation}

\subsection{Dataset}
We have evaluated our state of the art solution on a small dataset that we use for development. It is a collection of 50 Czech texts that are reporting on some accidents (car accidents and other actions of fire rescue services). These reports come from the web of Fire rescue service of Czech Republic\footnote{\url{http://www.hzscr.cz/hasicien/}}. The labeled corpus is publically available on the web of our project\footnote{\url{http://czsem.berlios.de/}}.
The corpus is structured such that each document represents one event (accident) and several attributes of the accident are marked in text. For the evaluation we selected two attributes of different kind. The first one is `damage' -- an amount (in CZK - Czech Crowns) of summarized damage arisen during a reported accident. The second one is `injuries', it marks mentions of people injured during an accident. These two attributes differ. Injuries annotations always cover only a single token, while damage annotations usually consist of two or three tokens -- one or two numerals express the amount and one extra token is for currency.

These two attributes differ in two directions:
\begin{enumerate}
	\item Injuries annotations always cover only a single token while damage usually consists of two or three tokens - one or two numerals express the amount and one extra token is for currency.
	\item The complexity of the marked information (and the difficulty of the corresponding extraction task) differs slightly. While labeling of all money amounts in the corpus will result in 75\% accuracy for damage annotations, in the case of injured persons mentions there are much more possibilities and indications are more spread in context.
\end{enumerate}

\subsection{Comparison with Paum classifier}
To compare our solution with other alternatives we took the Paum propositional learner from GATE \citep{Li:Paum}. The quality of propositional learning from texts is strongly dependent on the selection of right features. We obtained quite good results with features of a window of two preceding and two following token lemmas and morphological tags. The precision was further improved by adding the feature of \emph{analytical function} from the syntactic parser (see the last row of Table~\ref{tab:EvaluationResults}).

Because we did not want to invest much time to this and the feature setting of the Paum learner was quite simple (a window of two preceding and following token lemmas and morphological tags). We admit that looking for better features could further improve the results of the Paum learner.

\subsection{Cross validation}
We used the 10-fold cross validation in the evaluation. Thanks to this technique the evaluation is simple. After processing all the folds each document is processed with some of the ten learned models such that the particular document was not used in learning of that model, so all documents are unseen by the model applied on them. At the end we just compare the gold standard annotations with the learned ones in all documents.

\subsection{Results}


Results of a 10-fold cross validation are summarized in Table~\ref{tab:EvaluationResults}. We used standard information retrieval performance measures: precision, recall and $F_1$ measure and also theirs lenient variants (overlapping annotations are added to the correctly matching ones, the measures are the same if no overlapping annotations are present).

\begin{table}[t]
	\centering
			
\begin{tabular}{|l||r|r|r|r|r|r|r|}
\hline
\textbf{task/method} & \textbf{matching} & \textbf{missing} & \textbf{excessive} & \textbf{overlap} & \textbf{prec.}\% & \textbf{recall}\% & \textbf{F1.0}\%\\
\hline
\hline
\textbf{damage/ILP} & 14 & 0 & 7 & 6 & 51.85 & 70.00 & 59.57\\
\hline
\multicolumn{5}{|l|}{\textbf{damage/ILP -- lenient measures}} & 74.07 & 100.00 & 85.11\\
\hline
\textbf{dam./ILP-roots} & 16 & 4 & 2 & 0 & 88.89 & 80.00 & 84.21\\
\hline
\textbf{damage/Paum} & 20 & 0 & 6 & 0 & 76.92 & 100.00 & 86.96\\
\hline
\hline
\textbf{injuries/ILP} & 15 & 18 & 11 & 0 & 57.69 & 45.45 & 50.85\\
\hline
\textbf{injuries/Paum} & 25 & 8 & 54 & 0 & 31.65 & 75.76 & 44.64\\
\hline
\textbf{inj./Paum-afun} & 24 & 9 & 38 & 0 & 38.71 & 72.73 & 50.53\\
\hline
\end{tabular}
						
	\caption{Evaluation results }
	\label{tab:EvaluationResults}
	\vspace{-0.80cm}
\end{table}

In the first task (`damage') the methods obtained much higher scores then in the second (`injuries') because the second task is more difficult. In the first task also the root/subtree preprocessing/postprocessing improved results of ILP such that afterwards, annotation borders were all placed precisely. The ILP method had better precision and worse recall than the Paum learner but the $F_1$ score was very similar in both cases.

\subsubsection{Statistical Significance}
The term statistical significance refers to the result of a pair-wise comparison of learning engines using the corrected resampled (two tailed) T-Test \citep{Nadeau:2003:IGE:779909.779927}, which is suitable for cross validation based experiments. The Weka implementation\footnote{\url{http://www.cs.waikato.ac.nz/ml/weka/}} is used. Test significance is 0.05 in all cases.

\subsection{Examples of learned rules}

In Figure~\ref{fig:rules} we present some examples of the rules learned from the whole dataset. The rules demonstrate a connection of a target token with other parts of a sentence through linguistic syntax structures. For example the first rule connects a root numeral (\emph{n.quant.def}) of `damage' with a mention of `investigator' that stated the mount. In the last rule only a positive occurrence of the verb `injure' is allowed.

\begin{figure}
\begin{minted}[linenos,  fontsize=\footnotesize,
               frame=lines]{prolog}
%[Rule 1] [Pos cover = 14 Neg cover = 0]
damage_root(A) :- lex_rf(B,A), has_sempos(B,'n.quant.def'), tDependency(C,B),
   tDependency(C,D), has_t_lemma(D,'investigator').

%[Rule 2] [Pos cover = 13 Neg cover = 0]
damage_root(A) :- lex_rf(B,A), has_functor(B,'TOWH'), tDependency(C,B),
   tDependency(C,D), has_t_lemma(D,'damage').


%[Rule 1] [Pos cover = 7 Neg cover = 0]
injuries(A) :- lex_rf(B,A), has_functor(B,'PAT'), has_gender(B,anim),
   tDependency(B,C), has_t_lemma(C,'injured').

%[Rule 8] [Pos cover = 6 Neg cover = 0]
injuries(A) :- lex_rf(B,A), has_gender(B,anim), tDependency(C,B),
   has_t_lemma(C,'injure'), has_negation(C,neg0).
\end{minted}
	TODO: \verb+damage_root -> mention_root+
	\caption{Examples of learned rules, Czech words are translated.}
	\label{fig:rules}
\end{figure}





Experience with human-designed rules.










\section{Conclusion and Future Work}
From our experiments can be seen that ILP is capable to find complex and meaningful rules that cover the intended information. But in terms of the performance measures the results are not better than those from a propositional learner. This is quite surprising observation because Czech is a language with free word order and we would expect much better results of the dependency approach than those of the position based approach, which was used by the propositional learner.

Our method is still missing an intelligent semantic interpretation procedure and it should be evaluated on bigger datasets (e.g. MUC, ACE, TAC, CoNLL) and other languages. So far we also do not provide a method for classical relation extraction (like e.g. in \citep{Bunescu:DependencyPaths}). In the present solution we deal with relations implicitly. The method has to be adapted for explicit learning of relations in the form of ``subject predicate object''.

Our method can also provide a comparison of linguistic formalisms and tools because on the same data we could run our method using different linguistic analyzers and compare the results.
