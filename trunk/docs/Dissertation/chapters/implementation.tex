\chapter{Implementation} \label{sec:ch_implementation}


%%%%%%%%%%%%%%%%%%%%%%%%%%%%%%%%%%%%%%%%%%%%%%%%%%%%%%%%%%%%%%%%%%%%%%%%%%%%%%%%%%%%%
%%%%%%%%%%%%%%%%%%%%%%%%%%%%%%%%%%%%%%%%%%%%%%%%%%%%%%%%%%%%%%%%%%%%%%%%%%%%%%%%%%%%%
\section{Manual Design of Extraction Rules} \label{sec:impl_manual_rules} \graphicspath{{../img/ch50/}}


\subsection{Procedural Extraction Rules} \label{sec:ch50_Procedural_Extraction_Rules} \label{sec:impl_btred_rules}

Our first extraction method, which is based on procedural extraction rules, was implemented using Btred\footnote{See details about Btred in Section~\ref{sec:third_tred}.} API for processing linguistic trees. The extraction rules were implemented in Perl as Btred procedures. Application of these extraction rules on a corpus of linguistic trees is realized such that each procedure (or extraction rule) is executed on every available tree by Btred. 

An example of such extraction rule is in Figure~\ref{lst:btred_rule} and corresponding extraction output on Figure~\ref{fig:btred_xml}. TODO: explain


%%%%%%%%%%%%%%%%%%%%%%%%%%%%%%%%%%%%%%%%%%%%%%%%%%%%%%%%%%%%%%%%%%%%%%%%%%%%%%%%%%%%%
\begin{figure}
\begin{minted}[linenos,  fontsize=\footnotesize,
               frame=lines]{perl}
#variable $this contains currently processed node, $root current root node

my @injure_verbs = ("zranit", "usmrtit", "zemřít", "zahynout", "přežít");

sub print_injured {
	if ($this->{gram}{sempos} eq "v") {
		foreach my $v (@injure_verbs) {
			if ($this->{t_lemma} eq $v ) {
				#action type
				print "<action type=\"" . $this->{t_lemma} . "\">";

				#sentece
				print "<sentece>" . PML_T::GetSentenceString($root) . "</sentece>";
				print "<sentece_id>" . $root->{id} . "</sentece_id>";
				
				#negation
				if (test_negation($this)) {
					print "<negation>true</negation>" ;					
				} else {
					print "<negation>false</negation>" ;										
				}
								
				#manner of injurance
				my @mans = find_node_by_attr_depth($this, 0, 'functor', '^MANN');
				if (@mans) {
					foreach my $m (@mans) {
						print "<manner>"; 
						print $m->{t_lemma};
						print "</manner>"; 
					};
				}
				
				#actors and patients
				my @pats = find_node_by_attr($this, 'functor', '^[PA][AC]T');
				@pats = &filter_list(\&test_person, @pats);
				
				foreach my $p (@pats) {
					print "<participant type=\"" . $p->{t_lemma} . "\">";

					#patients count
					my @cnt = find_node_by_attr($p, 'functor', '^RSTR');
					@cnt = &filter_list(\&test_number_lemma, @cnt);
					my $cnt1 = pop(@cnt);
					print "<quantity>" . 
						&test_number($cnt1->{t_lemma}) . 
						"</quantity>" if ($cnt1);
	
					print "<full_string>";
					print_subtree_as_text($p);
					print "</full_string>";

					print "</participant>";
				}
				
				#action end
				print "</action>\n";											
}}}}
\end{minted} 
																																		\begin{comment}
																																		this >>$<< hacks my syntax highlighter :-)
																																		\end{comment}
\caption{Procedurally written extraction rule in \emph{Btred}.}
\label{lst:btred_rule}
\end{figure}
%%%%%%%%%%%%%%%%%%%%%%%%%%%%%%%%%%%%%%%%%%%%%%%%%%%%%%%%%%%%%%%%%%%%%%%%%%%%%%%%%%%%%




%%%%%%%%%%%%%%%%%%%%%%%%%%%%%%%%%%%%%%%%%%%%%%%%%%%%%%%%%%%%%%%%%%%%%%%%%%%%%%%%%%%%%
\begin{figure}
\begin{minted}[linenos,  fontsize=\footnotesize,
               frame=lines]{xml}
<injured_result>
	<action type="zranit">
		<sentece>
			Při požáru byla jedna osoba lehce zraněna -- jednalo se
			o majitele domu, který si vykloubil rameno.
		</sentece>
		<sentece_id>T-vysocina63466.txt-001-p1s4</sentece_id>
		<negation>false</negation>
		<manner>lehký</manner>
		<participant type="osoba">
			<quantity>1</quantity>
			<full_string>jedna osoba</full_string>
		</participant>
	</action>
	<action type="zemřít">
		<sentece>
			Ve zdemolovaném trabantu na místě zemřeli dva muži -- 82letý
			senior a další muž, jehož totožnost zjišťují policisté.
		</sentece>
		<sentece_id>T-jihomoravsky49640.txt-001-p1s4</sentece_id>
		<negation>false</negation>
		<participant type="muž">
			<quantity>2</quantity>
			<full_string>dva muži</full_string>
		</participant>
	</action>
		<action type="zranit">
		<sentece>čtyřiatřicetiletý řidič nebyl zraněn.</sentece>
		<sentece_id>T-jihomoravsky49736.txt-001-p4s3</sentece_id>
		<negation>true</negation>
		<participant type="řidič">
			<full_string>čtyřiatřicetiletý řidič</full_string>
		</participant>
	</action>
</injured_result>
\end{minted}
\caption{\emph{XML} structured output of the query written in \emph{Btred}.}
\label{fig:btred_xml}
\end{figure}
%%%%%%%%%%%%%%%%%%%%%%%%%%%%%%%%%%%%%%%%%%%%%%%%%%%%%%%%%%%%%%%%%%%%%%%%%%%%%%%%%%%%%





\subsection{Netgraph Based Extraction Rules} \label{sec:ch50_Netgraph_Based_Extraction_Rules}

Our second extraction method, which is based on Netgraph (see Section~\ref{sec:ch30_netgraph} for details), is implemented in Java. Java was chosen partly because we use the Java implementation of Netgraph client as a library. 



The extraction is implemented as follows: Netgraph implementation is responsible for the evaluation of the Netgraph query part of extraction rules and matching trees are then returned to our implementation, which prepares the extraction output based on the SELECT part of extraction rules.




\begin{figure}
	\centering
		\includegraphics[width=0.5\hsize]{eenv_extract_patern}
\\Transcript:\\
\begin{tabular}{|c|c|}
\hline
uniknout, unikat & vytéci\\
to leak out & to flow out\\
\hline
\end{tabular}		
	\caption{A manually created extraction rule investigating dangerous liquids that spilled out into the environment.}
	\label{fig:ch50_eenv_extract_patern}
\end{figure}


\begin{figure}
	\centering
		\includegraphics[angle=-90, width=0.6\hsize]{eenv_matching_tree}
		
Original sentence: 
\emph{``Nárazem se utrhl hrdlo palivové nádrže a do potoka postupně vyteklo na 800 litrů nafty.''}\\
English transcript: 
\emph{``Due to the clash the throat of fuel tank tore off and 800 liters of oil (diesel) has run out to a stream.''}
	\caption{A tree matching with the corresponding extraction rule in Figure~\ref{fig:ch50_eenv_extract_patern}.}
	\label{fig:ch50_eenv_matching_tree}
\end{figure}


%%%%%%%%%%%%%%%%%%%%%%%%%%%%%%%%%%%%%%%%%%%%%%%%%%%%%%%%%%%%%%%%%%%%%%%%%%%%%%%%%%%%%
\begin{figure}
\begin{minted}[linenos,  fontsize=\footnotesize,
               frame=lines]{xml}
<QueryMatches>
	<Match root_id="T-vysocina63466.txt-001-p1s4" match_string="2:0,7:3,8:4,11:2">
		<Sentence>
			Při požáru byla jedna osoba lehce zraněna - jednalo se
			o majitele domu, který si vykloubil rameno.
		</Sentence>
		<Data>
			<Value variable_name="action_type" attribute_name="t_lemma">zranit</Value>
			<Value variable_name="injury_manner" attribute_name="t_lemma">lehký</Value>
			<Value variable_name="participant" attribute_name="t_lemma">osoba</Value>
			<Value variable_name="quantity" attribute_name="t_lemma">jeden</Value>
		</Data>
	</Match>
	<Match root_id="T-jihomoravsky49640.txt-001-p1s4" match_string="1:0,13:3,14:4">
		<Sentence>
			Ve zdemolovaném trabantu na místě zemřeli dva muži - 82letý senior
			a další muž, jehož totožnost zjišťují policisté.
		</Sentence>
		<Data>
			<Value variable_name="action_type" attribute_name="t_lemma">zemřít</Value>
			<Value variable_name="participant" attribute_name="t_lemma">muž</Value>
			<Value variable_name="quantity" attribute_name="t_lemma">dva</Value>
		</Data>
	</Match>
	<Match root_id="T-jihomoravsky49736.txt-001-p4s3" match_string="1:0,3:3,7:1">
		<Sentence>Čtyřiatřicetiletý řidič nebyl zraněn.</Sentence>
		<Data>
			<Value variable_name="action_type" attribute_name="t_lemma">zranit</Value>
			<Value variable_name="a-negation" 
			       attribute_name="m/tag">VpYS---XR-NA---</Value>
			<Value variable_name="participant" attribute_name="t_lemma">řidič</Value>
		</Data>
	</Match>
</QueryMatches>
\end{minted}
\caption{\emph{XML} structured output of the SQL select like query. A negation can be detected from the presence of \emph{m/tag} on the line 30.}
\label{fig:select_xml}
\end{figure}
%%%%%%%%%%%%%%%%%%%%%%%%%%%%%%%%%%%%%%%%%%%%%%%%%%%%%%%%%%%%%%%%%%%%%%%%%%%%%%%%%%%%%

\begin{figure}
	\centering
		\includegraphics[angle=-90, width=0.7\hsize]{eenv_results}
	\caption{\emph{XML} structured output of the SQL select like query corresponding with the extraction rule in Figure~\ref{fig:ch50_eenv_extract_patern} and matching tree in Figure~\ref{fig:ch50_eenv_matching_tree}.}
	\label{fig:ch50_eenv_results}
\end{figure}























\subsubsection{Illustration Examples}



Evaluation of the extraction rule from Figure~\ref{fig:ch50_eenv_extract_patern} is illustrated on Figure~\ref{fig:ch50_eenv_matching_tree}. Figure~\ref{fig:ch50_eenv_matching_tree} shows a linguistic tree matching with the extraction rule. Matching nodes are decorated and labeled by the numbers of corresponding query nodes.



\subsection{Extraction Output} \label{sec:impl_manual_output}

Small pieces of extraction outputs are shown in Figure~\ref{fig:select_xml} (for the extraction rule in Figure~\ref{fig:ch50_extract_patern}) and in Figure~\ref{fig:ch50_eenv_results} (for the extraction rule in Figure~\ref{fig:ch50_eenv_extract_patern}). 

The former example (Figure~\ref{fig:select_xml}) contains three matches of the extraction rule in three different articles. Each query match is closed in the \verb+<Match>+ element and each contains values of some linguistic attributes closed inside the \verb+<Value>+ elements. Each value comes from some of the nodes of the extraction rule. Name of corresponding query node is saved in the \verb+variable_name+ attribute of the \verb+<Value>+ element.

In the case of the example query, values identified by the variable \verb+action_type+ specify the type of the action. So in the first and third case somebody was injured (\emph{zranit} means to injure in Czech, lines 8 and 28) and in the second case somebody died (\emph{zemřít} means to die in Czech, line 20).

Values identified by \verb+participant+ and \verb+quantity+ contain information about participants of the action. \verb+participant+ serves for specification of the type of the participants and \verb+quantity+ values hold numbers (quantity) of the participants. So in the first action one (\emph{jeden}, line 11) person (\emph{osoba}, line 10) was injured and in the second action two (\emph{dva}, line 22) men (\emph{muž}, line 21) died.

Values identified by \verb+a-negation+ contain the information about a negation of a clause (The presence of negation is indicated by the 11th character of the position-based morphological tag, note that the corresponding node (number 2) of the extraction rule is marked as optional and the restriction on m/tag is put in the form of regular expression on the 11th character.) So we can see that the participant (driver -- \emph{řidič}, line 31) of the last action was \textbf{not} injured (lines 29-30).

The last not described attribute name is \verb+injury_manner+. Corresponding values contain information about the manner of injury of an injury action. So in the first action of the example there was a light injury (\emph{lehký} means light in Czech, line 9).








%%%%%%%%%%%%%%%%%%%%%%%%%%%%%%%%%%%%%%%%%%%%%%%%%%%%%%%%%%%%%%%%%%%%%%%%%%%%%%%%%%%%%
%%%%%%%%%%%%%%%%%%%%%%%%%%%%%%%%%%%%%%%%%%%%%%%%%%%%%%%%%%%%%%%%%%%%%%%%%%%%%%%%%%%%%
\section{Machine Learning of Extraction Rules} \graphicspath{{../img/ch60/}} \label{sec:impl_learning}



Here we just briefly describe implementation of our system. The system consists of several modules, all integrated in GATE as processing resources (PRs).

\subsection{TectoMT Wrapper (Linguistic Analysis)} \label{sec:ch60_tectomt_wrapper}

TectoMT wrapper is a GATE component (processing resource), which takes the text of a GATE document, sends it to TectoMT linguistic analyzers, parses the results and converts the results to the form of GATE annotations. The next section provides details about how PDT annotations are represented in GATE.

Because TectoMT has to run as a separate process (it is implemented in Perl) and the initialization of TectoMT analyzers usually takes significant amount of time it would be very inefficient to start a new TectoMT instance for each document. Therefore the implementation currently offers two modes of execution: batch (TectoMTBatchAnalyser) and online (TectoMTOnlineAnalyser).

The batch mode is implemented similarly to the Batch Learning PR\footnote{\url{http://gate.ac.uk/userguide/sec:ml:batch-learning-pr}}. Batch mode of execution in the context of GATE corpus pipelines is a deviation from the standard execution mode, when documents are processed one by one. During the execution as a part of a corpus pipeline, TectoMTBatchAnalyser only accumulates documents and the whole work is done as a batch when the last document is encountered. This also implies that TectoMTBatchAnalyser has to be the last PR in the pipeline because it produces no output in the time of execution (except for the last document when the whole batch is executed). 

Client-server model of implementation is used in the online mode. A separate TectoMT server process is started at the time of initialization and during the execution, GATE documents are processed in ordinary way, one by one. This means that (similarly to the previous case) each document is converted to the TectoMT readable format, sent to TectoMT and the result is converted back to GATE. The online mode of execution is a bit slower than the batch mode because additional time is spent on client-server communication (XML-RPC\footnote{\url{http://www.xmlrpc.com/}}).



\subsection{PDT Annotations in GATE} \label{sec:ch60_pdt_in_gate} \label{sec:LDR_in_GATE}

Although GATE annotations are just highlighted pieces of text (see also Section~\ref{sec:ch30_gate_annotations}) it is possible to use them to encode dependency tree structures. It is possible because each GATE annotation has a unique identifier (ID) and an arbitrary set of features (name-value pairs) can be assigned to it. The way how the PDT dependency trees are encoded in GATE is in fact the same as in the GATE wrapper for the Stanford Parser\footnote{\url{http://gate.ac.uk/userguide/sec:parsers:stanford}}. 

Three main constructs are used to capture an arbitrary configuration of a linguistic dependency tree:


\begin{description}
	\item[tree nodes] (usually corresponding to words (tokens) of a sentence)
	\item[edges] (dependency relations between nodes)
	\item[node attributes] (connected linguistic features like POS, gender, tense, case, etc.)
\end{description}

These constructs are encoded in GATE in the following way: tree nodes correspond to \emph{token} annotations, node attributes are saved as token annotation features and edges are encoded as another special kind of annotations.

Two kinds of token annotations are used to represent two kinds of trees and tree nodes. ``Token'' annotation type is used for analytical tree nodes and ``tToken'' for tectogrammatical tree nodes.

Four kinds of edges (PDT dependencies) are implemented by the TectoMT wrapper: analytical dependencies, tectogrammatical dependencies, aux.rf (auxiliary reference) and lex.rf (main lexical reference). The last two kinds (aux.rf and lex.rf) are used to connect tectogrammatical and analytical nodes. The implementation differs according to the cardinality of a dependency type. The first three kinds are of the cardinality one-to-many (one parent node can have many children nodes) and the last one (lex.rf) if of the cardinality one-to-one (one parent node has at most one child). Because of that, lex.rf edges can be stored as features (with the name ``lex.rf'') of ``tToken'' annotations. Note that a single GATE annotation feature can have at most one value in each annotation. In this case the annotation ID of the referenced ``Token'' annotation (referenced analytical node) is the value of the lex.rf feature.

One-to-many dependencies are stored as separate annotations (type names: ``aDependency'', ``tDependency'', ``aux.rf'') with a single feature called ``args''. Values of this feature are of Java type List<Integer> (list of integers). The list always contains just two items. The first one is the annotation ID of the parent annotation; the second one is the ID of the child annotation. Instead of using one list feature, two simple features (like ``arg1'', ``arg2'' or ``parentID'', ``childID'') could be used, but the implementation is consistent with the wrapper for the Stanford Parser\footnote{\url{http://gate.ac.uk/userguide/sec:parsers:stanford}}, which uses the single list feature called ``args''), thus PDT dependencies are compatible with Stanford dependencies in GATE.

It is not simple to demonstrate the GATE representation of the dependencies in a static printed form; we can only show a GATE screenshot (Figure~\ref{fig:PDT_GATE}) that partly illustrates that.


\begin{figure}
	\centering
		\framebox{\includegraphics[width=0.7\hsize]{PDT_GATE}}
	\caption{PDT annotations in GATE (screenshot).}
	\label{fig:PDT_GATE}
\end{figure}


\subsubsection{Netgraph Tree Viewer} \label{sec:ch60_GATE_Netgraph}

Figure~\ref{fig:GATE_Netgraph}

\begin{figure}
	\centering
		\includegraphics[width=0.7\hsize]{netgraph_stanford}
		\\Sentence: Users also said it is in the strongest financial position in its 24-year history.
	\caption{Netgraph Tree Viewer in GATE (for Stanford Dependencies, screenshot).}
	\label{fig:GATE_Netgraph}
\end{figure}


\subsection{ILP Wrapper (Machine Learning)} \label{sec:impl_ilp_wrapper}
After a human annotator have annotated several documents with desired target annotations, machine learning takes place. 
This consists of two steps: 
\begin{enumerate}
	\item learning of extraction rules from the target annotations and
	\item application of the extraction rules on new documents.
\end{enumerate}
In both steps the linguistic analysis has to be done before and in both steps ILP background knowledge (a logical database of facts) is constructed from linguistic structures of documents that are being processed. We call the process of background knowledge construction as \emph{ILP serialization}; more details are presented below in Section~\ref{sec:ilp_serialization}.

After the ILP serialization is done, the next step depends on the phase which is being performed.

In the learning case, positive and negative examples are constructed from target annotations and the machine learning ILP inductive procedure is executed to obtain extraction rules.

In the application case a Prolog system is used to check if the extraction rules entail any of target annotation candidates.


%Learning / application
%\\1.	serialization -> learning in ILP
%\\2.	serialization -> application in ILP


The learning examples and annotation candidates are usually constructed from all document tokens (and we did so in the present solution), but it can be optionally changed to any other textual unit, for example only numerals or tectogrammatical nodes (words with lexical meaning) can be selected. This can be done easily with the help of \emph{Machine Learning PR} (LM PR) from GATE\footnote{\emph{Machine Learning PR} is an old GATE interface for ML and it is almost obsolete but in contrast to the new \emph{Batch Learning PR} the LM PR is easy to extend for a new ML engine.}.

ML PR provides an interface for exchange of features (including target class) between annotated texts and propositional learners in both directions -- during learning as well as during application. We have used ML PR and developed our \emph{ILP Wrapper} for it. The implementation was a little complicated because complex linguistic structures cannot be easily passed as propositional features, so in our solution we use the ML PR interface only for exchange of the class attribute and annotation id and we access the linguistic structures directly in a document.



\subsection{ILP Serialization} \label{sec:ilp_serialization} \label{sec:impl_ilp_serialization}

In this section details about conversion of linguistic trees to ILP background knowledge (a Prolog logical database of facts) will be presented. Although the construction is quite strait forward it is worth describing because it makes it easier to understand the extraction rules found by the ILP learning procedure. 

As mentioned in Section~\ref{sec:ch60_pdt_in_gate}: three main constructs are used to capture an arbitrary configuration of a dependency linguistic tree: nodes, edges and node attributes. During the process of ILP Serialization these constructs are rendered to Prolog in following way. 

A unique identifier (node ID) is generated for every tree node. The identifier is based on document name and GATE annotation ID.\footnote{Note that node IDs based on sentence order and node deep order are used outside of GATE, see PML$\rightarrow$RDF transformation in Section~\ref{sec:pml_to_rdf}.} These node IDs correspond to simple Prolog atoms and they represent tree nodes in the Prolog fact database. A node type (used by the ILP learning algorithm) is assigned to a node ID by predicates \texttt{Token(NodeID)} for analytical tree nodes and \texttt{tToken(NodeID)} for tectogrammatical tree nodes.

Tree nodes are connected by edges using several binary predicates with a common form:

\texttt{dependency\_type\_name(ParentNodeID, ChildNodeID)}

\noindent Note that the parent (governor) node always occupies the first argument and the child (dependant) node the second one. Predicate name \emph{tDependency} is used for tectogrammatical dependencies and \emph{aDependency} for analytical ones. There are also special kinds of dependencies that connect tectogrammatical and analytical nodes: \emph{lex.rf} (main lexical reference) and \emph{aux.rf} (auxiliary reference), in these cases tectogrammatical node occupies the first argument and analytical the second.

Node attributes are assigned to node IDs by binary predicates of the form:

\texttt{attribute\_name(NodeID, AttributeValue)}

\noindent There are about thirty such predicates like \emph{t\_lemma} (tectogrammatical lemma), \emph{functor} (tectogrammatical functor), \emph{sempos} (semantic part of speech), \emph{negation}, \emph{gender}, etc. but only some of them can be found in example extraction rules and we also excluded some of the attributes from serialization examples for space and simplicity reasons.

Example of a serialized tectogrammatical tree is in Figure~\ref{lst:ilp_serialization} it is the same tree as in Figure~\ref{img:ch10_damage_tree}.


%%%%%%%%%%%%%%%%%%%%%%%%%%%%%%%%%%%%%%%%%%%%%%%%%%%%%%%%%%%%%%%%%%%%%%%%%%%%%%%%%%%%%
\begin{figure}
\begin{minted}[linenos, fontsize=\footnotesize,
               frame=lines]{prolog}
tToken(  id_jihomoravsky47443_243).
t_lemma( id_jihomoravsky47443_243, 'být'). %to be
functor( id_jihomoravsky47443_243, 'PRED').
sempos(  id_jihomoravsky47443_243, 'v').
tDependency( id_jihomoravsky47443_243, id_jihomoravsky47443_238).
tToken(  id_jihomoravsky47443_238).
t_lemma( id_jihomoravsky47443_238, ','). %comma
functor( id_jihomoravsky47443_238, 'APPS').
sempos(  id_jihomoravsky47443_238, 'n.denot').
gender(  id_jihomoravsky47443_238, 'nr').
tDependency( id_jihomoravsky47443_238, id_jihomoravsky47443_237).
tToken(  id_jihomoravsky47443_237).
t_lemma( id_jihomoravsky47443_237, 'vyčíslit'). %to quantify
functor( id_jihomoravsky47443_237, 'PAT').
sempos(  id_jihomoravsky47443_237, 'v').
tDependency( id_jihomoravsky47443_237, id_jihomoravsky47443_245).
tToken(  id_jihomoravsky47443_245).
t_lemma( id_jihomoravsky47443_245, 'předběžně'). %preliminarily
functor( id_jihomoravsky47443_245, 'MANN').
sempos(  id_jihomoravsky47443_245, 'adv.denot.grad.nneg').
tDependency( id_jihomoravsky47443_237, id_jihomoravsky47443_244).
tToken(  id_jihomoravsky47443_244).
t_lemma( id_jihomoravsky47443_244, 'vyšetřovatel'). %investigator
functor( id_jihomoravsky47443_244, 'ACT').
sempos(  id_jihomoravsky47443_244, 'n.denot').
gender(  id_jihomoravsky47443_244, 'anim').
tDependency( id_jihomoravsky47443_237, id_jihomoravsky47443_240).
tToken(  id_jihomoravsky47443_240).
t_lemma( id_jihomoravsky47443_240, 'osm'). %eight
functor( id_jihomoravsky47443_240, 'PAT').
sempos(  id_jihomoravsky47443_240, 'n.quant.def').
gender(  id_jihomoravsky47443_240, 'nr').
tDependency( id_jihomoravsky47443_240, id_jihomoravsky47443_242).
tToken(  id_jihomoravsky47443_242).
t_lemma( id_jihomoravsky47443_242, 'tisíc'). %thousand
functor( id_jihomoravsky47443_242, 'RSTR').
sempos(  id_jihomoravsky47443_242, 'n.quant.def').
gender(  id_jihomoravsky47443_242, 'inan').
tDependency( id_jihomoravsky47443_242, id_jihomoravsky47443_247).
tToken(  id_jihomoravsky47443_247).
t_lemma( id_jihomoravsky47443_247, 'koruna'). %crown
functor( id_jihomoravsky47443_247, 'MAT').
sempos(  id_jihomoravsky47443_247, 'n.denot').
gender(  id_jihomoravsky47443_247, 'fem').
tDependency( id_jihomoravsky47443_237, id_jihomoravsky47443_246).
tToken(  id_jihomoravsky47443_246).
t_lemma( id_jihomoravsky47443_246, 'škoda'). %damage
functor( id_jihomoravsky47443_246, 'PAT').
sempos(  id_jihomoravsky47443_246, 'n.denot').
gender(  id_jihomoravsky47443_246,'fem').
\end{minted}
\caption{ILP serialization example}
\label{lst:ilp_serialization}
\end{figure}
%%%%%%%%%%%%%%%%%%%%%%%%%%%%%%%%%%%%%%%%%%%%%%%%%%%%%%%%%%%%%%%%%%%%%%%%%%%%%%%%%%%%%







\section{Shareable Extraction Ontologies}

\subsection{Data Transformation}

\subsection{Rule Transformations}

\section{Fuzzy ILP Classification}

\subsection{Translation of Fuzzy ILP Task to Several Classical ILP Tasks}

\subsection{Learned Rules Examples}
