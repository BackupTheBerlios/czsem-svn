\def\myauthor{Jan Dědek}
\def\mytitle{Semantic Annotations}
\def\mysupervisor{Prof. RNDr. Peter Vojtáš, DrSc.}
\def\mydepartment{Department of Software Engineering}

\hypersetup{pdftitle=\mytitle}
\hypersetup{pdfauthor=\myauthor}



%myemph	
\newcommand{\myemph}[1]{\textbf{\emph{#1}}}

%footurl
\newcommand{\footurl}[1]{\footnote{\url{#1}}}


%sectiondoubleref
\newcommand{\sectiondoubleref}[1]{\nameref{#1} (Section~\ref{#1})} 


%openright
\let\openright=\cleardoublepage

%myaddchaptertoc
\newcommand{\myaddchaptertoc}[1]{
\openright
\phantomsection
\addcontentsline{toc}{chapter}{#1}
}


%definition
\theoremstyle{definition}
\newtheorem{definition}{Definition}
\newtheorem{theorem}{Theorem}



% Tato makra přesvědčují mírně ošklivým trikem LaTeX, aby hlavičky kapitol
% sázel příčetněji a nevynechával nad nimi spoustu místa. Směle ignorujte.
\makeatletter
\def\@makechapterhead#1{
  {\parindent \z@ \raggedright \normalfont
   \Huge\bfseries \thechapter. #1
   \par\nobreak
   \vskip 20\p@
}}
\def\@makeschapterhead#1{
  {\parindent \z@ \raggedright \normalfont
   \Huge\bfseries #1
   \par\nobreak
   \vskip 20\p@
}}
\makeatother


% Toto makro definuje kapitolu, která není očíslovaná, ale je uvedena v obsahu.
\def\chapwithtoc#1{
\chapter*{#1}
\addcontentsline{toc}{chapter}{#1}
}
