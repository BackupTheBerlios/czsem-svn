% Pokud tiskneme oboustranně:
 \documentclass[12pt,a4paper,twoside,openright]{report}
\usepackage[a4paper, 
						twoside, 
						inner=2.9cm, 
						outer=1.9cm, 
						bottom=2.6cm, 
						top=2.1cm] {geometry}
 
 
\let\openright=\cleardoublepage

\def\myauthor{Jan Dědek}
\def\mytitle{Semantic Annotations}
\def\mysupervisor{Prof. RNDr. Peter Vojtáš, DrSc.}
\def\mydepartment{Department of Software Engineering}

\usepackage{nameref}
\usepackage[square]{natbib}
\usepackage{url}
\usepackage{nomencl}
\usepackage{multicol}
\usepackage{array}
\usepackage{amsthm}
\usepackage{verbatim}



%minted staci \usepackage[utf8]{inputenc}
\usepackage{minted}

\usepackage{bigfoot}
\usepackage{hyperref}
\hypersetup{pdftitle=\mytitle}
\hypersetup{pdfauthor=\myauthor}


\usepackage[xetex]{graphicx}

\usepackage{rotating}
\usepackage{longtable}
\usepackage{multirow}

\usepackage{xltxtra}
\setsansfont{Calibri}
\setmonofont{Consolas}

\makenomenclature 



% Tato makra přesvědčují mírně ošklivým trikem LaTeX, aby hlavičky kapitol
% sázel příčetněji a nevynechával nad nimi spoustu místa. Směle ignorujte.
\makeatletter
\def\@makechapterhead#1{
  {\parindent \z@ \raggedright \normalfont
   \Huge\bfseries \thechapter. #1
   \par\nobreak
   \vskip 20\p@
}}
\def\@makeschapterhead#1{
  {\parindent \z@ \raggedright \normalfont
   \Huge\bfseries #1
   \par\nobreak
   \vskip 20\p@
}}
\makeatother

% Toto makro definuje kapitolu, která není očíslovaná, ale je uvedena v obsahu.
\def\chapwithtoc#1{
\chapter*{#1}
\addcontentsline{toc}{chapter}{#1}
}


\begin{document}

% Trochu volnější nastavení dělení slov, než je default.
\lefthyphenmin=2
\righthyphenmin=2

\pagestyle{empty}
\begin{center}

\large

Charles University in Prague

\medskip

Faculty of Mathematics and Physics

\vfill

{\bf\Large DOCTORAL THESIS}

\vfill

\centerline{\mbox{\includegraphics[width=60mm]{style/logo}}}

\vfill
\vspace{5mm}

{\LARGE \myauthor}

\vspace{15mm}

% Název práce přesně podle zadání
{\LARGE\bfseries \mytitle}

\vfill

% Název katedry nebo ústavu, kde byla práce oficiálně zadána
% (dle Organizační struktury MFF UK)
\mydepartment

\vfill

\begin{tabular}{rl}

Supervisor of the doctoral thesis: & \mysupervisor \\
\noalign{\vspace{2mm}}
Study programme: & Softwarové systémy \\
%\noalign{\vspace{2mm}}
%Specialization: & specialization \\
\end{tabular}

\vfill

% Zde doplňte rok
Prague 2012

\end{center}

\newpage

%%% Na tomto místě mohou být napsána případná poděkování (vedoucímu práce,
%%% konzultantovi, tomu, kdo zapůjčil software, literaturu apod.)

\chapter*{Acknowledgments}
I wish to thank my supervisor ...

\newpage

%%% Strana s čestným prohlášením k disertační práci

\vglue 0pt plus 1fill

\noindent
I declare that I carried out this doctoral thesis independently, and only with the cited
sources, literature and other professional sources.

\medskip\noindent
I understand that my work relates to the rights and obligations under the Act No.
121/2000 Coll., the Copyright Act, as amended, in particular the fact that the Charles
University in Prague has the right to conclude a license agreement on the use of this
work as a school work pursuant to Section 60 paragraph 1 of the Copyright Act.

\vspace{10mm}

\hbox{\hbox to 0.5\hsize{%
In ........ date ............
\hss}\hbox to 0.5\hsize{%
signature of the author
\hss}}

\vspace{20mm}
\newpage

%%% Povinná informační strana disertační práce

\vbox to 0.5\vsize{
\setlength\parindent{0mm}
\setlength\parskip{5mm}

Název práce:
Sémantické Anotace
% přesně dle zadání

Autor:
\myauthor

Katedra:  % Případně Ústav:
Katedra Softwarového Inženýrství
% dle Organizační struktury MFF UK

Vedoucí disertační práce:
\mysupervisor
% dle Organizační struktury MFF UK, případně plný název pracoviště mimo MFF UK

Abstrakt:
% abstrakt v rozsahu 80-200 slov; nejedná se však o opis zadání disertační práce

Klíčová slova:
% 3 až 5 klíčových slov

\vss}\nobreak\vbox to 0.49\vsize{
\setlength\parindent{0mm}
\setlength\parskip{5mm}

Title:
\mytitle

Author:
\myauthor

Department:
\mydepartment
% dle Organizační struktury MFF UK v angličtině

Supervisor:
\mysupervisor
% dle Organizační struktury MFF UK, případně plný název pracoviště
% mimo MFF UK v angličtině

Abstract:
% abstrakt v rozsahu 80-200 slov v angličtině; nejedná se však o překlad
% zadání disertační práce

Keywords:
% 3 až 5 klíčových slov v angličtině

\vss}

\newpage

%%% Strana s automaticky generovaným obsahem disertační práce. U matematických
%%% prací je přípustné, aby seznam tabulek a zkratek, existují-li, byl umístěn
%%% na začátku práce, místo na jejím konci.

\openright

%vykopat cislovani
\addtocontents{toc}{\protect\thispagestyle{empty}}

\tableofcontents
%\addcontentsline{toc}{chapter}{Contents}


\chapwithtoc{Introduction}
\setcounter{page}{1}










%%% Jednotlivé kapitoly práce jsou pro přehlednost uloženy v samostatných souborech
\chapter{Extraction Method Based on ILP Machine Learning} \label{ch:ILP_Learning}
\graphicspath{{../img/ch60/}}


\section{Introduction}
Automated semantic annotation (SA) is considered to be one of the most important elements in the evolution of the Semantic Web. Besides that, SA can provide great help in the process of data and information integration and it could also be a basis for intelligent search and navigation.


A description, implementation and initial evaluation of the method are the main contributions of the present work.







%\section{Data Flow ?or Schema of the Extraction Process?}








\section{Evaluation} %\label{sec:learning_eval}


For the evaluation we selected two attributes of different kind. The first one is `damage' -- an amount (in CZK - Czech Crowns) of summarized damage arisen during a reported accident. The second one is `injuries', it marks mentions of people injured during an accident. These two attributes differ. Injuries annotations always cover only a single token, while damage annotations usually consist of two or three tokens -- one or two numerals express the amount and one extra token is for currency.

These two attributes differ in two directions:
\begin{enumerate}
	\item Injuries annotations always cover only a single token while damage usually consists of two or three tokens - one or two numerals express the amount and one extra token is for currency.
	\item The complexity of the marked information (and the difficulty of the corresponding extraction task) differs slightly. While labeling of all money amounts in the corpus will result in 75\% accuracy for damage annotations, in the case of injured persons mentions there are much more possibilities and indications are more spread in context.
\end{enumerate}


\subsection{Results}


Results of a 10-fold cross validation are summarized in Table~\ref{tab:EvaluationResults}. We used standard information retrieval performance measures: precision, recall and $F_1$ measure and also theirs lenient variants (overlapping annotations are added to the correctly matching ones, the measures are the same if no overlapping annotations are present).

\begin{table}[t]
	\centering
			
\begin{tabular}{|l||r|r|r|r|r|r|r|}
\hline
\textbf{task/method} & \textbf{matching} & \textbf{missing} & \textbf{excessive} & \textbf{overlap} & \textbf{prec.}\% & \textbf{recall}\% & \textbf{F1.0}\%\\
\hline
\hline
\textbf{damage/ILP} & 14 & 0 & 7 & 6 & 51.85 & 70.00 & 59.57\\
\hline
\multicolumn{5}{|l|}{\textbf{damage/ILP -- lenient measures}} & 74.07 & 100.00 & 85.11\\
\hline
\textbf{dam./ILP-roots} & 16 & 4 & 2 & 0 & 88.89 & 80.00 & 84.21\\
\hline
\textbf{damage/Paum} & 20 & 0 & 6 & 0 & 76.92 & 100.00 & 86.96\\
\hline
\hline
\textbf{injuries/ILP} & 15 & 18 & 11 & 0 & 57.69 & 45.45 & 50.85\\
\hline
\textbf{injuries/Paum} & 25 & 8 & 54 & 0 & 31.65 & 75.76 & 44.64\\
\hline
\textbf{inj./Paum-afun} & 24 & 9 & 38 & 0 & 38.71 & 72.73 & 50.53\\
\hline
\end{tabular}
						
	\caption{Evaluation results }
	\label{tab:EvaluationResults}
	\vspace{-0.80cm}
\end{table}

In the first task (`damage') the methods obtained much higher scores then in the second (`injuries') because the second task is more difficult. In the first task also the root/subtree preprocessing/postprocessing improved results of ILP such that afterwards, annotation borders were all placed precisely. The ILP method had better precision and worse recall than the Paum learner but the $F_1$ score was very similar in both cases.




















\chapwithtoc{Conclusion}


%%% Seznam použité literatury
\bibliographystyle{../style/iisproc}
\bibliography{../dedek_phd}
\addcontentsline{toc}{chapter}{Bibliography}


%\chapwithtoc{List of Figures}
\listoffigures
\addcontentsline{toc}{chapter}{List of Figures}


%%% Tabulky v disertační práci, existují-li.
%\chapwithtoc{List of Tables}
\listoftables
\addcontentsline{toc}{chapter}{List of Tables}

%%% Použité zkratky v disertační práci, existují-li, včetně jejich vysvětlení.
%\chapwithtoc{List of Abbreviations}
\nomenclature{API}{Application Programming Interface}

\nomenclature{GPL}{GNU General Public License \\\url{http://www.gnu.org/licenses/gpl.html}}

\nomenclature{GRDDL}{Gleaning Resource Descriptions from Dialects of Languages \\\url{http://www.w3.org/TR/grddl/}}

\nomenclature{IE}{Information Extraction}

\nomenclature{ILP}{Inductive Logic Programming}

\nomenclature{JAPE}{Java Annotation Patterns Engine \\\url{http://gate.ac.uk/userguide/chap:jape}}

\nomenclature{ML}{Machine Learning}

\nomenclature{OWL}{Web Ontology Language \\\url{http://www.w3.org/TR/owl-primer/}}

\nomenclature{PAUM}{Perceptron Algorithm with Uneven Margins}

\nomenclature{PML}{Prague Markup Language}

\nomenclature{PDT}{Prague Dependency Treebank}

\nomenclature{RDFa}{Resource Description Framework - in - attributes \\\url{http://www.w3.org/TR/xhtml-rdfa-primer/}}

\nomenclature{RDF}{Resource Description Framework \\\url{http://www.w3.org/RDF/}}

\nomenclature{RPC}{Remote Procedure Call}

\nomenclature{RSS}{Really Simple Syndication, or Rich Site Summary, but originally RDF Site Summary \\\url{http://www.rssboard.org/rss-specification}}

\nomenclature{SPARQL}{SPARQL Query Language for RDF \\\url{http://www.w3.org/TR/rdf-sparql-query/}}

\nomenclature{SQL}{Structured Query Language \\\url{http://www.iso.org/iso/catalogue_detail.htm?csnumber=45498}}

\nomenclature{XML}{Extensible Markup Language \\\url{http://www.w3.org/XML/}}

\nomenclature{XSLT}{Extensible Stylesheet Language Transformations \\\url{http://www.w3.org/TR/xslt}}


\printnomenclature[2.5cm]
\addcontentsline{toc}{chapter}{Nomenclature}

%%% Přílohy k disertační práci, existují-li (různé dodatky jako výpisy programů,
%%% diagramy apod.). Každá příloha musí být alespoň jednou odkazována z vlastního
%%% textu práce. Přílohy se číslují.
\chapwithtoc{Attachments}

\openright
\end{document}
